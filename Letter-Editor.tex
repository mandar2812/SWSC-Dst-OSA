\documentclass{letter}
\usepackage{hyperref}
\signature{Mandar Chandorkar \\ Enrico Camporeale \\ Simon Wing}
%\address{21 Bridge Street \\ Smallville \\ Dunwich DU3 4WE}
\begin{document}

\begin{letter}{Editor \\ Journal of Space Weather and Space Climate}
\opening{Dear Dr. Consuelo Cid:}

We are submitting the revised version of the manuscript, where all the points raised by the referees have been carefully taken into account.

However, we would like to point out that the main comment by referee \#9 revolves around a paragraph in the cited paper by Ji et al. (2012), that we believe has led the referee to a complete misunderstanding of the approach used in that paper.

Of course, this is an important point to clarify given that our results are based on the tests performed by Ji et al. The referee argues that Ji et al. have tested the models not by using the standard one-hour-ahead approach, but rather by propagating the Dst values predicted by the models. This approach would lead to degraded (i.e. less than optimal) performance, and we do not understand what would be the rationale of doing so, in a paper whose first objective is to compare the accuracy of different models.

In any case, to completely clear up any doubt about this important concern, we have tested the NARMAX model (which is one of the best performing models) using both approaches (real data or model predicted data for Dst). The result is shown in the reply to the referees, and it shows that the Ji et al. paper could not have possibly used the approach suggested by the referee. Indeed, not using the real Dst data severely degrades the performance of NARMAX.
We hope that this would settle the argument and, therefore,  that the paper will be deemed publishable in the current form.

\closing{Yours Faithfully,}


\end{letter}
\end{document}